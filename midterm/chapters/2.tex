\vspace{-1cm}
\section*{جزئیات و تاریخچه پیاده‌سازی}
\vspace{-0.2cm}
\vspace{-0.4cm}
\subsection*{تاریخچه مختصر}
\vspace{-0.15cm}
اولین نکته‌ای که در مورد این ویژگی حائز اهمیت است، این است که از حدود سال $2012$ زمزمه‌های پیاده‌سازی آن وجود داشته ولی به دلایل حواشی فراوان پیرامون محدود کردن سطح دسترسی کاربران و همچنین بحث پیرامون تاثیرگذار یا تاثیرگذار نبودن آن، این موضوع به نسخه نهایی هسته راه‌نیافته بود؛ اما ویژگی‌های مشابهی در توزیع‌های مختلف لینوکس برای عملکردی مشابه پیاده‌سازی شده بودند.
\cite{LWN2}

یکی از دلایل این حواشی مربوط به این می‌شود که در هسته لینوکس، مکانیزمی برای اعمال سیاست‌های امنیتی در قالب 
\lr{Linux Security Module}
قرار گرفته بود. این مکانیزم دست کاربر را برای اعمال محدودیت‌های امنیتی مختلف باز می‌گذارد. با این وجود مشکلی که وجود دارد، این بخش از فضای کاربر قابل اعمال است اما چنین سازوکاری که قرار است جلوی ویرایش نابه‌جای هسته را بگیرد، باید پیش از این که بتوان سیاست امنیتی خاصی را اعمال کرد، فعال بشود.
\vspace{-0.9cm}
\subsection*{جزئیات پیاده‌سازی }
\vspace{-0.35cm}
در نتیجه مواردی که در قسمت قبل گفته شد، نوعی ماژول امنیتی مقدماتی در هنگام بالاآمدن
\LTRfootnote{Boot}
سیستم ایجاد می‌شود تا قلاب‌های امنیتی
\LTRfootnote{Security Hook}
اولیه لازم برای مکانیزم قفل کردن هسته ثبت بشوند. این قلاب امنیتی در پرونده 
\lr{\Verb+lsm\_hooks.h+}
تعریف شده و در پرونده 
\lr{\Verb+security.c+}
در تابع 
\lr{\Verb+security\_locked\_down+}
فراخوانی می‌شود.
\vspace{0.1cm}
\begin{code}
//lsm_hooks.h
1820: int (*locked_down)(enum lockdown_reason what);
	
//security.c	
2402: int security_locked_down(enum lockdown_reason what)
{
	return call_int_hook(locked_down, 0, what);
}
\end{code}

عملا ایجاد این قلاب امنیتی به هسته اجازه می‌دهد حالت‌های مختلف نامعتبر بودن را که در توابع مختلف امنیتی مشخص شده‌اند، به درستی چک کند.

فعال‌سازی حالت قفل‌سازی هسته به شکلی است که علاوه بر این که بعد از اجرا شدن می‌توان آن را فعال کرد
\cite{man}
، در هنگام بالا آمدن
 هم می‌توان آن را به عنوان پارامتر مشخص کرد. همچنین در هنگام ساختن
 \LTRfootnote{Build}
 کد هسته هم می‌توان آن را در پرونده‌های
 \lr{\Verb+KConfig+}
 مشخص کرد.
 
 به طور کلی سه حالت برای قفل‌سازی هسته وجود دارد. \lr{\Verb+LOCKDOWN\_NONE+} که عملا این مورد را فعال نمی‌کند. حالت 
 \lr{\Verb+LOCKDOWN\_INTEGRITY\_MAX+}
 که جلوی عملیات‌هایی که منجر به تغییر غیرمجاز در هسته (عموما از سمت کاربر ریشه) می‌شوند را گرفته و حالت
 \lr{\Verb+LOCKDOWN\_CONFIDENTIALITY\_MAX+}
 که حتی امکان افشای داده‌های سطح هسته را هم می‌گیرد. با این وجود باید توجه کرد که در اصل حالت‌های خیلی بیش‌تری وجود دارند که همه آن‌ها در قالب  \lr{\Verb+enum lockdown\_reason +} در پرونده \lr{\Verb+Security.h+} قرار گرفته‌اند. به دلیل ترتیبی بودن \lr{\Verb+enum+}ها، عملا فعلا‌سازی 
 \lr{\Verb+LOCKDOWN\_INTEGRITY\_MAX+}
 باعث فعال‌ شدن همه حالت‌های امنیتی پیشین آن نظیر
 \lr{\Verb+LOCKDOWN\_KEXEC+}
 هم می‌شود.
 \cite{kcm}
 
 به عنوان مثال یکی از تغییراتی که در این نسخه داده شده است، مربوط به پرونده 
 \lr{\Verb+kexec\_file.c+}
 است:
 
 \begin{code}
 	static int kimage_validate_signature(struct kimage *image){
 	...
	 	if (!ima_appraise_signature(READING_KEXEC_IMAGE) &&
			security_locked_down(LOCKDOWN_KEXEC))
	 		return -EPERM;
 	...	
 }
 \end{code}
این تابع در صورتی که امضای پرونده توسط واحد \lr{\Verb+ima+}
\LTRfootnote{Integrity Measurement Architecture}
تایید نشود و حالت
\lr{\Verb+LOCKDOWN\_KEXEC+}
هم فعال باشد، خطای مجاز نبودن عملیات (\lr{\Verb+EPERM+})
 می‌دهد. در این جا لازم است به تغییراتی که در پرونده 
 \lr{\Verb+ima\_policy.c+}
 داده‌ شده است هم اشاره کنیم. در تابع
 \lr{\Verb+ima\_appraise\_signature+}
 تابعی که شناسه مربوط به آن داده شده است را با لیست قوانین موجود در \lr{\Verb+ima+} مقایسه می‌کند و در صورتی که یکی از این قواعد، تابعی مشابه تابع داده شده داشته و امضای دیجیتال آن معتبر بود، نتیجه را به صورت 
 \lr{\Verb+true+}
 باز می‌گرداند.
 \cite{linux}
 
 \begin{code}
 	bool ima_appraise_signature(enum kernel_read_file_id id)
 	{ ...
 		func = read_idmap[id] ?: FILE_CHECK;
 		list_for_each_entry_rcu(entry, ima_rules, list) {
 			if (entry->action != APPRAISE) continue;
 			if (entry->func && entry->func != func)	continue;
 			if (entry->flags & IMA_DIGSIG_REQUIRED) found = true;
 			break;
 		}
 	return found; }
 \end{code}
 
 بخش عمده تغییرات مربوط به این قابلیت مربوط به پرونده \lr{\Verb+lockdown.c+} می‌شود. مثلا پیاده‌سازی اصلی تابع نظیر شده به قلاب امنیتی
  \lr{\Verb+locked\_down+}
  با نام 
  \lr{\Verb+lockdown\_is\_locked\_down+}
  در این پرونده قرار دارد. در این پرونده توابعی برای خواندن و نوشتن هم پیاده‌سازی شده است که در اصل یکسری بررسی‌ها روی وضعیت و حالات قفل‌سازی هسته انجام داده و در صورت نیاز پیام‌ها یا خطاهایی را که در هنگام خروجی‌دادن این توابع باید تولید بشوند، مشخص کرده و در نهایت عملکرد اصلی خواندن و نوشتن را انجام می‌دهند. \cite{linux}
  
  همچنین تابعی برای فعال‌سازی وضعیت قفل‌کردن هسته هم به نام 
  \lr{\Verb+lock\_kernel\_down+}
  وجود دارد که در آن کنترل می‌شود که سطح خواسته شده  برای قفل کردن کمتر از سطح فعلی نباشد و در صورتی که مشکلی نبود، سطح قفل‌سازی هسته به سطح گفته شده ارتقا می‌یابد. قسمت‌های جالی از کد پیرامون نگاشت قلاب‌های امنیتی و چک صورت گرفته در هنگام قفل‌کردن کرنل برای سطح دسترسی مشخص که در این پرونده قرار دارند، در زیر آورده شده است:
  
  \begin{code}
 static int lock_kernel_down(const char *where, enum lockdown_reason level)
  { ...
  	if (kernel_locked_down >= level)
  		return -EPERM;
  ... }
  
	static struct security_hook_list lockdown_hooks[] __lsm_ro_after_init={
		LSM_HOOK_INIT(locked_down, lockdown_is_locked_down),};
  \end{code}

بدین ترتیب به شکلی نسبتا سطح بالا، نحوه پیاده‌سازی قابلیت قفل‌کردن هسته را در لینوکس بررسی کردیم. این قابلیت ریزه‌کاری‌های دیگری هم در قسمت‌های دیگر کد هسته دارد. تغییراتی جزئی در کدهای مربوط به 
\lr{\Verb+ACPI+}
\LTRfootnote{Advanced Configuration and Power Interface}
که در هنگام بوت‌شدن سیستم برای مدیریت توان و شناسایی قطعات سخت‌افزاری کاربرد دارد داده شده است. به علاوه تغییراتی جزئی در 
\lr{\Verb+file.c+}
و
\lr{\Verb+inode.c+}
داده شده تا وضعیت قفل‌بودن هسته در حالاتی خاص گزارش بشود. از ذکر ریز این جزئیات به دلیل این که در اصل عملکرد نقش مهمی نداشتند، چشم‌پوشی شده است. 
\cite{linux}