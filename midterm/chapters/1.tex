\vspace{-0.4cm}
\section*{معرفی و  انگیزه ایجاد این ویژگی}
\vspace{-0.15cm}
قفل‌کردن هسته
\LTRfootnote{Kernel Lockdown}
یکی از ویژگی‌های جدید هسته سیستم‌عامل لینوکس
\LTRfootnote{Linux Kernel}
است که در نسخه \lr{5.4} به آن اضافه شده است.
هدف از ایجاد این ویژگی، قراردادن مرز مجکم‌تری بین بین فضای کاربر
\LTRfootnote{User space}
و فضای هسته
\LTRfootnote{Kernel Space} 
از طریق محدودسازی کاربر ریشه
\LTRfootnote{root}
است.
\cite{zdnet}
 کاربر ریشه که با \lr{\Verb+UID 0+} مشخص می‌شود، به طور پیش‌فرض سطح دسترسی بالایی به تمام سیستم داشته و حتی امکان ویرایش هسته را هم دارد.
\cite{kenreln}

فناوری‌هایی نظیر \lr{UEFI Secure Boot} هم به همین منظور ایجاد شده‌اند تا اطمینان حاصل کنند که یک سیستم قفل شده، تنها برنامه‌هایی را اجرا می‌کند که توسط منبعی معتبر امضا شده باشد. با این وجود، از آن جایی که کاربر ریشه امکان ویرایش کد هسته را دارد، عملا نمی‌توان چنین عملکرد درستی را تضمین کرد و اگر کنترل کاربر ریشه از دست صاحب اصلی سیستم خارج شده باشد یا کد مخربی را اجرا کند، امنیت سیستم به خطر می‌افتد.
\cite{LWN}

مکانیزم قفل‌کردن هسته از همین رو ایجاد شده است تا در صورت فعال‌سازی در سیستم‌هایی که امنیت آن‌ها اهمیت بالایی دارد، بخشی از دسترسی‌های کاربر ریشه را هم غیرفعال کرده و اجازه تغییرات غیرمجاز در کدهای هسته را ندهد. بدین ترتیب مرز بین فضای کاربر و فضای هسته مستحکم‌تر شده و در سیستم‌هایی که این مکانیزم در آن‌ها فعال باشد، نمی‌توان برخی از تغییرات را حتی به کمک کاربر ریشه انجام داد.
\cite{LWN}
